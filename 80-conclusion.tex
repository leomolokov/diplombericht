\Conclusion % заключение к отчёту

В соответствии с целью и задачами выпускной квалификационной работы получены следующие результаты:

\begin{enumerate}[1)]
	\item Проверены, сведены в одной работе, а также, в целях, возникших при разработке алгоритмов и кода ПО, выведены  математические зависимости между параметром выпуклости (\textit{bulge}) и радиусом, углом раствора, координатами концов, координатами центра, и другими геометрическими параметрами дуги окружности;
	\item Разработаны алгоритмы конвертации формата DXF в форматы TXT(DXF-type), TXT(x,y,r), SVG, JSON по определённым правилам;
	\item Разработаны алгоритмы по нахождению геометрических параметров чертежа в DXF (список всех координат, габариты чертежа, наименьшие координаты), а также объектов DXF (радиус сегмента полилинии, центр дуги, радиус дуги по параметру \textit{bulge});
	\item Разработаны конвертеры из формата DXF в форматы TXT(DXF-type), TXT(x,y,r), SVG, JSON в виде программного обеспечения <<primiview>> на ЯП Python:
	\begin{enumerate}[4.1)]
		\item Разработана архитектура внутренней репрезентации примитивов DXF в программе <<primiview>>;
		\item Разработаны модули визуализации поддерживаемый объектов DXF, конвертации форматов;
		\item Разработан пользовательский интерфейс с помощью библиотеки PyQt;
		\item Проведено ручное тестирование программного обеспечения, в ходе которого была подтверждена корректность конвертации выбранных типов объектов DXF в указанные форматы.
	\end{enumerate}
	\item Проведён экономический расчёт:
	\begin{enumerate}[5.1)]
		\item Разработана экономическая модель проекта;
		\item Построено дерево задач, необходимых для выполнения целей проекта;
		\item Построена диаграмма Ганта и сетевой график,  в результате которых проект спланирован для реализации за 230 календарных дней;
		\item Проведён анализ сравнительной экономической эффективности проекта.
	\end{enumerate}
\end{enumerate}

\paragraph{Перспективы дальнейшей работы над проектом} 
\nopagebreak

Можно выделить следующие направления дальнейшего развития и совершенствования программного обеспечения <<primiview>> по конвертации форматов для обработки геометрической информации 2D-объектов:

\begin{enumerate}[1)]
	\item Расширение поддерживаемых для чтения и конвертации объектов DXF-файлов, например поддержка сплайнов, эллипсов, надписей;
	\item Создание поддержки распознования и конвертации приложением толщины и цветов линий;
	\item Расширение поддерживаемых входных форматов для конвертации, например Autodesk-форматов: DWG, DWS, DWT; Компас3D-форматов: FRW, FRT, CDW, CDT; DSS SolidWorks-форматов: DRW, SLDDRW, DRWDOT;
	\item Создание новых конвертеров для преобразования DXF, например в формат языка разметки YAML. Полезным и пока ещё нереализованным может быть конвертация во фрагменты формата TEX на языке создания графики в \TeX-среде --- Tikz. Также, удобными для дальнейшей работы форматами, в которые необходима конвертация DXF могут быть форматы векторной и растровой графики: PDF, GIF, PNG, JPEG.
\end{enumerate}


%%% Local Variables: 
%%% mode: latex
%%% TeX-master: "rpz"
%%% End: 
