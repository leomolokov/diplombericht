\Introduction

\paragraph{Актуальность темы исследования.} В настоящее время в рамках проектов Уральского Федерального Университета (УрФУ) и Уральского межрегионального научно-образовательного центра (УМНОЦ) происходит разработка специализированных САПР. Создаваемые ПП работают с файлами, содержащими геометрическую 2D-информацию. Используются следующие форматы файлов:

\begin{enumerate}
	\item DXF,
	\item TXT,
	\item SVG,
	\item JSON.
\end{enumerate}

В связи с этим требуется разработать ПО по конвертации данных форматов файлов.

\paragraph{Степень разработанности темы исследования.} В открытом доступе сети Интернет существуют онлайн-конвертеры файлов DXF в другие форматы. Большая их часть лишь визуализирует графическую информацию, представленную в том или ином DXF-файле, и конвертирует в наиболее популярные форматы изображений (JPEG, PNG). Для использования конвертера на предприятии-заказчике необходим особый формат данных, в которые конвертируется DXF. Такие форматы не могут быть обеспечены существующим в открытом рынке ПО --- таким, как, например, <<DXF Reader GT>> от компании «Gray Technical». Несмотря на высокую степень проработки программы, формат выходных данных не соответствует требованиям, предъявляемым к TXT-, SVG- и JSON-файлов, компании <<Unimatic>>. Ещё одним недостатком зарубежного ПО является ежемесячная плата разработчикам. Одной из задач современной Российской Федерации является импортозамещение на производстве, поэтому разработка собственного ПП увеличит независимость и самостоятельность предприятий и компаний, пользующихся этим ПП.

\paragraph{Цель работы} заключается в разработке программного обеспечения для дешифрования информации из файлов типа DXF с геометрической информацией объектов специального типа, извлечении из них информации о геометрических примитивах, вывода полученной графической информации на экран для верификации, формирования текстового файла (TXT) с выводом данных в исходном виде DXF, формирования текстового файла (TXT) с выводом данных в виде координат точек и радиуса примитивов (линий, дуг) между ними, формирования файла-описание двумерной векторной графики в формате SVG с выводом данных в исходном виде DXF, формирования текстового файла (JSON) с выводом данных в виде координат точек и степенями кривизны примитивов между ними.

Данные конвертеры необходимы, в частности, при разработке таких САПР, как Сириус, ТокКТЭ и других.

Для достижения этой цели поставлены следующие \textbf{задачи}:

\begin{enumerate}[1)]
	\item проанализировать входные данные DXF, в соответствии с разрабатываемым ПО;
	\item выявить структурно-содержательные особенности файлов формата DXF для последующей работы с разрабатываемым ПО;
	\item проанализировать возможности применения разных ЯП для разработки ПО;
	\item разработать алгоритмы;
	\item разработать ПО;
	\item провести анализ экономической целесообразности разрабатываемого проекта (сравнительная экономическая эффективность).
\end{enumerate}

\textbf{Объект исследования} --- формат файлов обмена графической информацией DXF.

\textbf{Предмет исследования} --- проектирование ПО для конвертации файлов и формата DXF в форматы TXT, SVG, JSON.

\paragraph{Теоретическая и практическая значимость работы}

\begin{enumerate}
	\item исследование универсально, то есть подход к разработке, используемый в данной работе, может быть реализован для создания ПП по конвертированию в другие файловые форматы;
	\item разработанное ПО можно внедрять в существующие САПР, работающие с форматом файлов DXF;
	\item САПР, использующие разработанный модуль по конвертации файлов с графической информацией, получают независимость от использования аналогичного ПО зарубежного производства;
	\item положенные в основу алгоритмы и написанное ПО имеет открытый доступ в сети Интернет и имеет перспективу развития в полноценный модуль импорта/экспорта файлов;
	\item коммерческая выгода заказчика.
\end{enumerate}