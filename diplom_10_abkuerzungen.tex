\Abbreviations %% Список обозначений и сокращений в тексте

В настоящей ВКР применяют следующие сокращения и обозначения:

\begin{description}
	
\item[САПР] Cистема автоматизированного проектирования.

\item[УП] Управляющая программа.

\item[ЧПУ] Числовое программное управление.

\item[ПО] Программное обеспечение.

\item[ПП] Программный продукт.

\item[ЯП] Язык программирования.

\item[DXF] Drawing eXchange Format. Формат файлов для обмена графической информацией между приложениями САПР, созданный фирмой Autodesk для системы AutoCAD в 1982 г.

\item[TXT] Текстовый формат файлов, представляющий собой последовательность строк электронного текста.

\item[SVG] Scalable Vector Graphics. Язык разметки масштабируемой векторной 2D-графики на основе XML, поддерживающий интерактивность и анимацию. Разрабатывается консорциумом World Wide Web с 1999 года по сегодняшний день.

\item[XML] eXtensible Markup Language. Расширяемый язык разметки подобный HTML.

\item[JSON] JavaScript Object Notation. Текстовый формат обмена данными, основанный на языке программирования JavaScript.

\end{description}

%%% Local Variables:
%%% mode: latex
%%% TeX-master: "rpz"
%%% End:
