\sloppy

\usepackage{G2-105}

\usepackage[utf8x]{inputenc}

% Настройки стиля ГОСТ 7-32
% Для начала определяем, хотим мы или нет, чтобы рисунки и таблицы нумеровались в пределах раздела, или нам нужна сквозная нумерация.
\EqInChapter % формулы будут нумероваться в пределах раздела
\TableInChapter % таблицы будут нумероваться в пределах раздела
\PicInChapter % рисунки будут нумероваться в пределах раздела

% Добавляем гипертекстовое оглавление в PDF
\usepackage[
bookmarks=true, colorlinks=true, unicode=true,
urlcolor=black,linkcolor=black, anchorcolor=black,
citecolor=black, menucolor=black, filecolor=black,
]{hyperref}

% Изменение начертания шрифта --- после чего выглядит таймсоподобно.
% apt-get install scalable-cyrfonts-tex

\IfFileExists{cyrtimes.sty}
    {
        \usepackage{cyrtimespatched}
    }
    {
        % А если Times нету, то будет CM...
    }

\usepackage{graphicx}   % Пакет для включения рисунков

% С такими оно полями оно работает по-умолчанию:
% \RequirePackage[left=20mm,right=10mm,top=20mm,bottom=20mm,headsep=0pt]{geometry}
% Если вас тошнит от поля в 10мм --- увеличивайте до 20-ти, ну и про переплёт не забывайте:
\geometry{right=20mm}
\geometry{left=30mm}


% Пакет Tikz
\usepackage{tikz}
\usetikzlibrary{arrows,positioning,shadows}

% Произвольная нумерация списков.
\usepackage{enumerate}

% ячейки в несколько строчек
\usepackage{multirow}

% itemize внутри tabular
\usepackage{paralist,array}

%счётчик
\makeatletter
\def\lastnumeq@putlabel{\addtocounter{page}{-1}
	\immediate\write\@auxout{\string
		\newlabel{LastNumberedEquation}{{\arabic{equation}}{\thepage}}}%
}
\AtEndDocument{%
	\clearpage\lastnumeq@putlabel}%
\makeatother


%positioning the picture at the exact place with [H]
\usepackage{float}


%algos packages
%\usepackage{algorithm2e}
\usepackage[linesnumbered,boxed]{algorithm2e}

%\usepackage{varwidth}

% Перевод плагина
\SetKwInput{KwData}{Исходные данные}
\SetKwInput{KwResult}{Результат}
\SetKwInput{KwIn}{Входные данные}
\SetKwInput{KwOut}{Выходные данные}
\SetKwIF{If}{ElseIf}{Else}{если}{тогда}{{иначе если}}{иначе}{{конец условия}}
\SetKwFor{While}{{до тех пор, пока}}{выполнять}{{конец цикла}}
\SetKw{KwTo}{от}
\SetKw{KwRet}{возвратить}
\SetKw{Return}{возвратить}
\SetKwBlock{Begin}{{начало блока}}{{конец блока}}
\SetKwSwitch{Switch}{Case}{Other}{{Проверить значение}}{{и выполнить}}{вариант}{{в противном случае}}{{конец варианта}}{{конец проверки значений}}
\SetKwFor{For}{цикл}{выполнять}{{конец цикла}}
\SetKwFor{ForEach}{{для каждого}}{выполнять}{{конец цикла}}
\SetKwRepeat{Repeat}{повторять}{{до тех пор, пока}}
\SetAlgorithmName{Алгоритм}{алгоритм}{Список алгоритмов}
\SetAlgoCaptionLayout{centerline}
\SetAlgoCaptionSeparator{ --- }

%package to do plots
\usepackage{pgfplots}

\usepackage{tabularray}
\DefTblrTemplate{contfoot-text}{default}{Продолжение на след. стр.}
\DefTblrTemplate{conthead-text}{default}{Продолжение таблицы}

\usepackage{listings}
\definecolor{maroon}{rgb}{0.5,0,0}
\definecolor{darkgreen}{rgb}{0,0.5,0}
\lstdefinelanguage{XML}
{
%	basicstyle=\ttfamily,
	basicstyle=\small,
	morestring=[s]{"}{"},
	morecomment=[s]{?}{?},
	morecomment=[s]{!--}{--},
	commentstyle=\color{darkgreen},
	moredelim=[s][\color{black}]{>}{<},
	moredelim=[s][\color{red}]{\ }{=},
	stringstyle=\color{blue},
	identifierstyle=\color{maroon}
}

\usepackage{xcolor}

\colorlet{punct}{red!60!black}
\definecolor{background}{HTML}{EEEEEE}
\definecolor{delim}{RGB}{20,105,176}
\colorlet{numb}{magenta!60!black}

\lstdefinelanguage{json}{
	basicstyle=\normalfont\ttfamily,
	numbers=left,
	numberstyle=\scriptsize,
	stepnumber=1,
	numbersep=8pt,
	showstringspaces=false,
	breaklines=true,
	literate=
	*{0}{{{\color{numb}0}}}{1}
	{1}{{{\color{numb}1}}}{1}
	{2}{{{\color{numb}2}}}{1}
	{3}{{{\color{numb}3}}}{1}
	{4}{{{\color{numb}4}}}{1}
	{5}{{{\color{numb}5}}}{1}
	{6}{{{\color{numb}6}}}{1}
	{7}{{{\color{numb}7}}}{1}
	{8}{{{\color{numb}8}}}{1}
	{9}{{{\color{numb}9}}}{1}
	{:}{{{\color{punct}{:}}}}{1}
	{,}{{{\color{punct}{,}}}}{1}
	{\{}{{{\color{delim}{\{}}}}{1}
	{\}}{{{\color{delim}{\}}}}}{1}
	{[}{{{\color{delim}{[}}}}{1}
	{]}{{{\color{delim}{]}}}}{1},
}

\usepackage{typearea}
\usepackage{pdflscape}
\usepackage{rotating}
\usepackage{lscape}

\usepackage{pdfpages}


\usepackage{totcount}
\usepackage[figure,table]{totalcount}

\regtotcounter{page}
% \regtotcounter{chapter}
\regtotcounter{totalcount@figure}
\regtotcounter{totalcount@table}
\newtotcounter{totalappendix}
\newtotcounter{totalchapter}
%%http://www.linux.org.ru/forum/general/6993203#comment-6994589 (используется totcount)
\makeatletter
\def\formbytotal#1#2#3#4#5{%
	\newcount\@c
	\@c\totvalue{#1}\relax
	\newcount\@last
	\newcount\@pnul
	\@last\@c\relax
	\divide\@last 10
	\@pnul\@last\relax
	\divide\@pnul 10
	\multiply\@pnul-10
	\advance\@pnul\@last
	\multiply\@last-10
	\advance\@last\@c
	\total{#1}~#2%
	\ifnum\@pnul=1#5\else%
	\ifcase\@last#5\or#3\or#4\or#4\or#4\else#5\fi
	\fi
}
\makeatother

