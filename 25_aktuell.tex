\chapter{Задача конвертации данных из файлов графической информации DXF. Анализ текущего состояния проблемы исследования}
\label{cha:aktuell}

В данном разделе описаны теоретические основы, требующиеся для разработки алгоритмов и программного обеспечения для обработки геометрической информации 2D-объектов специального типа.

По причине существования различных форматов для работы с одними данными, есть необходимость \textbf{конвертации} этих форматов друг в друга.

Конвертацией данных называется преобразование данных из одного формата в другой с сохранением основного логически-структурного содержания информации.

Конвертация может производиться как с потерей информации, так и без неё. Потеря информации может производиться намеренно при условном определении поддерживаемой разрабатываемым конвертером информации файлов.

Для исключения потерь необходимой установленной информации при конвертации данных, необходимо изучить форматы, из которых и в которые производится конвертация данных.

\section{Стандартизованные файловые форматы обмена графической информацией}

CAD-, CAM- и CAE-системы работают, в основном, помимо прочего, с графической информацией с целью визуализации данных. При этом по мере развития различных САПР от всевозможных организаций по всему миру для каждой системы создавались и создаются специальные форматы данных, с которыми удобно работать той или иной системе.

AutoCAD от компании Autodesk, являясь одной из первых в мире широкоиспользуемых CAD-систем, разработала норму файлового формата для работы с чертежами --- DXF.

Кроме этого, разработчики специализированных САПР до сих пор создают всё новые форматы, в которых работает именно их система. 



В рамках работы были изучены 5 форматов файлов, работающих с геометрической информацией. Среди них следующие:
\begin{itemize}
	\item DXF,
	\item TXT(DXF-type),
	\item TXT(x,y,r),
	\item SVG,
	\item JSON.
\end{itemize}

Работа ориентирована именно на эти форматы данных, что подробно объясняется в разделе  \ref{sec:actual}.

Рассмотрим детально каждый из стандартизованных форматов данных.

\subsection{Формат DXF} \label{subsec:allaboutdxf}

\paragraph{Краткая характеристика}

\begin{longtable}{p{110pt} p{340pt}}
%	\caption{}
	\label{tab:dxf}
	\centering
	\textbf{Название}:&AutoCAD DXF\\
	\textbf{Также известен как}:&AutoCAD Drawing Interchange Format, DXF, .DXB, .SLD, .ADI\\
	\textbf{Тип данных}:&Векторный\\
	\textbf{Сжатие}:&нет\\
	\textbf{Максимальный размер изображения}:&неограниченный\\
	\textbf{Несколько изображений в одном файле}:&нет\\
	\textbf{Разработчик}:&Autodesk\\
	\textbf{Поддерживающ. приложения}:&AutoCAD, различные САПР, CorelDraw, др.\\
	\textbf{Использование}:&Хранение и обмен САПР-(проектировочными) и векторными данными\\
	\textbf{Комментарии}:&Сложный формат, в основном, потому что может содержать в себе много различных типов данных. Формат разработан и поддерживается компанией Autodesk с целью применения в CAD-системе AutoCAD. Самая распространённая форма DXF --- 7-битный текст, однако существует, также, два схожих двоичных формата, один из которых представляется в виде расширения DXF, а другой --- DXB.\\
\end{longtable}

\paragraph{Обзор}

Форматы AutoCAD DXF (Drawing Interchange Format) и AutoCAD DXB (Drawing Interchange Binary) связаны с CAD-системой AutoCAD, созданной и поддерживаемой Autodesk. DXB — это упрощенная двоичная версия файла DXF. Другими форматами файлов, связанными с AutoCAD, являются форматы слайдов (.SLD) и графиков (.ADI).

Несмотря на то, что DXF был разработан для представления данных в САПР, он используется многими другими программами как формат обмена многими различными типами данных, чаще всего векторно-ориентированной информацией, а также текстом и 3D-полигонами. Как формат САПР, он также может выражать общие концепции черчения, такие как ассоциативные размеры.

Почти любой тип данных может быть каким-либо образом представлен в DXF. Например, программа для рисования CorelDraw! экспортирует контуры чертежа с объектом AutoCAD POLYLINE, в то время как 3D-программа может экспортировать только объекты 3DFACE, представляющие трёх- и четырёхсторонние многоугольники. DXF, также, позволяет создавать множество способов делать почти одно и то же, например, описывать объекты как отдельные редактируемые группы. Одна программа может размещать объекты на разных слоях рисунка, в то время как другая может использовать разные цвета пера, а третья может использовать именованные «блоки» для группировки данных.

Хоть DXF и широко используется для обмена простыми линейными данными, разработчик приложений, желающий поддерживать в них DXF, должен учитывать, что AutoCAD может хранить эти многочисленные типы данных различными способами.

Иногда правильная интерпретация файла DXF может быть очень сложной. Предполагаемый внешний вид линий и областей может зависеть от многих, казалось бы, непонятных настроек в заголовке DXF-файла. Поскольку файлы DXF очень сложно правильно интерпретировать, многие разработчики приложений решают экспортировать только DXF.

Даже среди программ, заявляющих об импорте DXF, можно обнаружить, что они поддерживают лишь часть всего, что возможно в DXF. Если есть необходимость создать свои собственные файлы DXF для передачи данных в программу, которая утверждает, что импортирует DXF, нужно убедиться, что известно, какие представления она понимает.

С каждой новой версией AutoCAD, DXF изменяется. AutoCAD версии 13 расширил формат DXF во многих отношениях, чтобы представить специализированные данные нового механизма геометрии. Эти дополнения хранят информацию о сложных поверхностях и твёрдых телах для геометрического механизма ACIS компании Spatial Technology, который теперь является частью AutoCAD. Не вся эта информация была задокументирована и должна быть пропущена любым читателем DXF. В версии 13 собственный допуск AutoCAD для числовых файлов DXF также изменился, поскольку он расширил шаг аудита, который проверяет достоверность импортируемых файлов DXF.

Очевидно, что формат файла DXF довольно сложный и тонкий. Далее приведена базовая структура любого файла DXF.

\paragraph{Организация файла}

Файл DXF состоит из семи разделов: заголовка, таблиц, блоков, классов, объектов, сущностей и маркера конца файла.

\begin{itemize}
	\item Раздел HEADER содержит переменные, представляющие состояние внутренних настроек AutoCAD. Например, для переменной версии AutoCAD «\$ACADVER» установлено значение «AC1012» в файле DXF, сохраненном AutoCAD версии 13. Другие переменные задают единицы измерения углов, значения по умолчанию для снятия фаски, смещения, масштабирования и т. д.
	\item Раздел TABLES содержит несколько массивов информации, используемой в остальной части чертежа, например список типов линий, имён слоев, шрифтов и предустановленных видов чертежа.
	\item Раздел BLOCKS содержит предопределенные элементы чертежа, которые могут присутствовать на чертеже. Например, блок может определять стандартную канавку, которая размещается на каждой секции вала определённого диаметра на чертеже. На определения блоков ссылаются в разделе ENTITIES с помощью команды INSERT.
	\item Разделы CLASSES и OBJECTS были представлены начиная с AutoCAD версии 13. Раздел CLASSES содержит описание любых определяемых приложением классов объектов, которые могут быть реализованы в разделах BLOCKS или ENTITIES.
	\item Раздел OBJECTS содержит неграфические части чертежа. Все сущности, которые не являются частью сущностей или таблиц символов, являются «объектами». Например, здесь хранятся словари AutoCAD.
	\item Раздел ENTITIES содержит фактические данные объекта чертежа. Сюда могут входить необработанные данные, такие как объекты LINE и ARC, а также команды INSERT, которые помещают предопределенное определение блока в определенную позицию на чертеже.
	\item Конец данных DXF отмечается директивой EOF в последней строке файла.
\end{itemize}

\paragraph{Детали файла}
Файл DXF состоит из пар групповых кодов и связанных значений. Каждое из них находится в отдельной строке текстового файла. Целочисленный групповой код указывает тип значения, за которым следует. Групповые коды встречаются в диапазонах. Например, за групповыми кодами от 0 до 9 следуют строки, и каждый отдельный групповой код используется в определенных случаях. Групповой код 0 указывает на начало объекта, таблицы или индикатора конца файла. Код 1 указывает основное текстовое значение объекта. Код группы 2 используется для имён, таких как имена разделов, блоков, имен таблиц и т. д. Код 9 вводит имя переменной раздела заголовка. Например, в начале каждого файла DXF код группы 0 предшествует команде SECTION, за которой следует код группы 2 со строкой, указывающей тип раздела, например HEADER:

\begin{lstlisting}[label=list:dxfheader]
0
SECTION
2
HEADER
9
$ACADVER
1
AC1012
\end{lstlisting}

Диапазоны групповых кодов указывают тип данных, которым следует следовать. Групповые коды от 10 до 59 используются для значений с плавающей точкой, таких как координаты точек. Коды с 60 по 79 хранят целочисленные значения. Например, для сохранения местоположения 2D-точки сначала используется групповой код 10 для значения X, затем код 20 используется для значения Y. Если объект имеет вторичное значение координаты, он также будет использовать групповые коды 11 и 21. Вот минимальный, но полный файл DXF, который описывает линию от точки (1,2) до (3,4) в плоском пространстве:

\begin{lstlisting}[label=list:dxflinefull]
0
SECTION
2
ENTITIES
999
This is just a line
0
LINE
8
0
10
1.0
20
2.0
11
3.0 21
4.0 0
ENDSEC
0
EOF
\end{lstlisting}

Код группы 999 предшествует комментарию. Эта строка будет помещена на слой 0, на что указывает групповой код 8. Этот минимальный файл является примером файла «только объекты», который будет принят практически любой программой, которая утверждает, что импортирует DXF.

Поскольку AutoCAD расширяется с каждой новой версией, добавляются новые групповые коды. При написании программы, которая читает файлы DXF, можно обеспечить совместимость в будущем, игнорируя неопределенные пары кода группы и значения.
Одним любопытным аспектом DXF является то, что он не содержит цветовой палитры, однако большинству объектов в файле DXF можно присвоить отдельное значение цвета с групповым кодом 62. Каждому объекту чертежа может быть присвоен номер от 1 до 255, известный как AutoCAD Color Index, или ACI, также описанный в более ранней документации как «номер пера». Это отражает происхождение AutoCAD как пакета САПР, в котором чертежи обычно печатались на перьевом плоттере с несколькими чернильными перьями, но без стандартного соответствия фактическим значениям RGB или даже цветам линий на экране. AutoCAD теперь устанавливает цвет RGB по умолчанию для каждого ACI, когда он появляется на экране, но они не сохраняются в файле DXF \cite{murray1996encyclopedia}.

\subsection{Формат SVG} \label{subs:svg}
\paragraph{Краткая характеристика}

\begin{longtable}{p{110pt} p{340pt}}
	%	\caption{}
	\label{tab:svg}
	\centering
	\textbf{Название}:&image/svg+xml\\
	\textbf{Также известен как}:&SVG, SVGZ (изображения, сжатые с помощью gzip)\\
	\textbf{Тип данных}:&Векторный/растровый\\
	\textbf{Сжатие}:&SVGZ\\
	\textbf{Максимальный размер изображения}:&неограниченный\\
	\textbf{Несколько изображений в одном файле}:&нет\\
	\textbf{Разработчик}:&World Wide Web Consortium (SVG Working Group)\\
	\textbf{Поддерживающ. приложения}:&браузеры, редакторы изображений, инструменты и библиотеки\\
	\textbf{Использование}:&Доступный обзор и обмен векторными изображениями, построение графиков, сложные элементы пользовательского интерфейса, логотипы, простые игры\\
	\textbf{Комментарии}:&Язык для описания двумерной графики в XML, созданный Консорциумом Всемирной паутины (W3C). Поддерживает как неподвижную, так и анимированную интерактивную графику — или, в иных терминах, декларативную и скриптовую. Не поддерживает описания трёхмерных объектов.\\
\end{longtable}

\paragraph{Обзор}

SVG - это язык для описания двумерной графики в XML [XML10, XML11]. SVG позволяет создавать графические объекты трех типов: векторные графические фигуры (например, контуры, состоящие из прямых линий и кривых), мультимедиа (растровые изображения, видео и аудио) и текст.

Документы SVG могут быть интерактивными и динамическими. Анимации могут быть определены и запущены либо декларативно (то есть путем встраивания анимационных элементов SVG в содержимое SVG), либо с помощью сценариев.

Достоинством формата SVG можно назвать, среди прочего, его представление в текстовом формате, то есть файлы SVG можно читать и редактировать при помощи обычных текстовых редакторов. При просмотре документов, содержащих SVG-графику, имеется доступ к просмотру кода просматриваемого файла и возможность сохранения всего документа. Кроме того, SVG-файлы обычно получаются меньше по размеру, чем сравнимые по качеству изображения в форматах JPEG (Joint Photographic Experts Group) или GIF (Graphics Interchange Format ), а также хорошо поддаются сжатию.

Масштабируемость изображения в формате SVG подразумевает возомжность увеличить любую часть изображения SVG без потери качества.

Широко доступно использование растровой графики в SVG-документах позволяет вставлять элементы с изображениями в форматах PNG, GIF или JPG.

Текст в графике SVG является текстом, а не изображением, поэтому его можно выделять и копировать, он индексируется поисковыми машинами, для этого не требуется создавать дополнительные метафайлы для поисковых роботов.

Анимация реализована в SVG с помощью языка SMIL (Synchronized Multimedia Integration Language), разработанного также консорциумом W3C. Поддерживаются скриптовые языки на основе спецификации ECMAScript. SVG-элементами можно управлять с помощью JavaScript. Применение скриптов и анимации в SVG позволяет создавать динамичную и интерактивную графику. В SVG обеспечивается событийная модель, отслеживаются события (загрузка страницы, изменение её параметров, события мыши, клавиатуры и др.). Анимация может запускаться по определённому событию (например «onmouseover» или «onclick»), что придаёт графике интерактивность. У каждого элемента есть свои собственные события, к которым можно привязывать отдельные скрипты.

SVG — открытый стандарт. В отличие от некоторых других форматов, SVG не является чьей-либо собственностью.

SVG-документы легко интегрируются с HTML-(HyperText Markup Language — «язык гипертекстовой разметки») и XHTML-(ExtensibleHTML)документами. Внешние SVG подключаются через тег object, значение атрибута data — имя файла с расширением «.svg», содержащего разметку SVG, и имеющего MIME-тип image/svg+xml. Атрибуты width и height определяют размеры области SVG по горизонтали и по вертикали. Элементы SVG совместимы с HTML и DHTML (Dynamic HTML).

SVG предоставляет все преимущества XML:
\begin{itemize}
	\item Возможность работы в различных средах;
	\item Интернационализация (поддержка Юникода);
	\item Широкая доступность для различных приложений;
	\item Лёгкая модификация через стандартные API — например, DOM (Document Object Model). SVG поддерживает стандартизированную W3C объектную модель документа DOM, обеспечивая доступ к любому элементу, что даёт широкие возможности по динамическому изменению элементов, их атрибутов и событий.;
	\item Лёгкое преобразование таблицами стилей XSLT. Как любой основанный на XML формат, SVG даёт возможность использовать для его обработки таблицы трансформации (XSLT). Преобразуя XML-данные в SVG с помощью простого XSL, можно легко получить графическое представление любых данных, например визуализировать химические молекулы, описанных на языке CML.
\end{itemize}

\paragraph{Организация файла}
\nopagebreak

Фрагмент документа SVG состоит из любого количества элементов SVG.

Фрагмент документа SVG может варьироваться от пустого фрагмента (т. е. без содержимого внутри, <<svg>> элемент), до очень простого фрагмента документа SVG, содержащего один SVG графический элемент, такой как "прямоугольник".

Фрагмент документа SVG может существовать сама по себе как автономный файл или ресурс, в этом случае Фрагмент документа SVG является документом SVG, или он может быть встроен в качестве фрагмента в родительский XML-документ.

В следующем примере показано простое содержимое SVG, встроенное в виде фрагмента в родительский XML-документ. Стоит обратить внимание на использование пространств имён XML, чтобы указать, что элементы <<svg>> и <<эллипс>> принадлежат пространству имён SVG:
\begin{lstlisting}[language=XML,label=list:xml]
<?xml version="1.0"?>
<parent xmlns="http://example.org"
  xmlns:svg="http://www.w3.org/2000/svg">
  <!-- parent contents here -->
  <svg:svg width="4cm" height="8cm" version="1.2" baseProfile="tiny" viewBox="0 0 100 100">
  <svg:ellipse cx="50" cy="50" rx="40" ry="20" />
  </svg:svg>
  <!-- ... -->
</parent>
\end{lstlisting}

Фрагмент документа SVG может содержать только один <<svg>> элемент. Это означает, что <<svg>> элементы не могут находиться внутри другого SVG.

В любом случае, для соответствия либо пространству имён в XML 1.0, либо пространству имён в XML 1.1 согласно Рекомендациям [XML-NS10, XML-NS], объявление пространства имён SVG должно находиться в области видимости для элемента <<svg>>, так чтобы все элементы SVG идентифицировались, как принадлежащие к пространству имён SVG.

Например, атрибут <<xmlns>> без префикса может быть указан на <<svg>> элемент, что означает, что SVG является пространством имён по умолчанию для всех элементов в пределах области действия элемента с атрибутом <<xmlns>>:
\begin{lstlisting}[language=XML,label=list:xml]
	<?xml version="1.0"?>
	<svg xmlns="http://www.w3.org/2000/svg" version="1.2" baseProfile="tiny">
	  <desc>Demonstrates use of a default namespace prefix for elements.</desc>
	  <rect width="7" height="3"/>
	</svg>
\end{lstlisting}

Если префикс пространства имён указан на атрибуте <<xmlns>> (например, xmlns:svg="http://www.w3.org/2000/svg") тогда соответствующее пространство имён не будет являться пространством имён по умолчанию, поэтому элементам должен быть присвоен явный префикс пространства имён \cite{andersson2008scalable}:
\begin{lstlisting}[language=XML,label=list:xml]
	<?xml version="1.0"?>
	<s:svg xmlns:s="http://www.w3.org/2000/svg" version="1.2" baseProfile="tiny">
	  <s:desc>Demonstrates use of a namespace prefix for elements.
	  Notice that attributes are not namespaced</s:desc>
	<s:rect width="7" height="3"/>
	</s:svg>
\end{lstlisting}

%\pagebreak
\subsection{Формат JSON} \label{subs:json}
\paragraph{Краткая характеристика}
%\nopagebreak

\begin{longtable}{p{110pt} p{340pt}}
	%	\caption{}
	\label{tab:svg}
	\centering
	\textbf{Название}:&json\\
	\textbf{Также известен как}:&JSON, JavaScript Object Notation\\
	\textbf{Тип данных}:&Текстовый\\
	\textbf{Разработчик}:&Дуглас Крокфорд\\
	\textbf{Поддерживающ. приложения}:&текстовые редакторы/обзорщики, специализированное ПО\\
	\textbf{Использование}:&API, Web-сервисы, обмен большим объёмом данных\\
	\textbf{Комментарии}:&Простой формат обмена данными, удобный для чтения и написания как человеком, так и компьютером.\\
\end{longtable}

\paragraph{Обзор.}
JSON основан на подмножестве языка программирования JavaScript, определенного в стандарте ECMA-262 3rd Edition - December 1999. JSON - текстовый формат, полностью независимый от языка реализации, но он использует соглашения, знакомые программистам C-подобных языков, таких как C, C++, C\#, Java, JavaScript, Perl, Python и многих других. Эти свойства делают JSON идеальным языком обмена данными.

JSON основан на двух структурах данных:
\begin{enumerate}[1)]
	\item Коллекция пар ключ/значение. В разных языках, эта концепция реализована как объект, запись, структура, словарь, хэш, именованный список или ассоциативный массив, 
	\item Упорядоченный список значений. В большинстве языков это реализовано как массив, вектор, список или последовательность.
\end{enumerate}

Это универсальные структуры данных. Почти все современные языки программирования поддерживают их в какой-либо форме. Логично предположить, что формат данных, независимый от языка программирования, должен быть основан на этих структурах.

\paragraph{Организация файла.} Следующий пример показывает JSON-представление данных об объекте, описывающем человека. В данных присутствуют строковые поля имени и фамилии, информация об адресе и массив, содержащий список телефонов. Как видно из примера, значение может представлять собой вложенную структуру.
\begin{lstlisting}[language=json,label=list:json]
{
	"firstName": "Ivan",
	"lastName": "Ivanov",
	"address": {
		"streetAddress": "Moskovskoe sh., 101, kv.101",
		"city": "Leningrad",
		"postalCode": 101101
	},
	"phoneNumbers": [
	"812 123-1234",
	"916 123-4567"
	]
}
\end{lstlisting}

\section{Способы описания геометрии дуг} \label{sec:arcs}

Данный раздел приведён в работе, так как дуги являются, исходя из опыта разработки алгоритмов и ПО <<primiview>>, самыми сложными объектами для обработки из всех поддерживаемых геометрических примитивов.

Дуги можно описывать различными способами. Разработчики САПР ищут наиболее выгодные из них с целью удобства дальнейшей работы с примитивами.

В AutoCAD для описания дуг используются следующие параметры:
\begin{itemize}
	\item координата центра по $Ox$,
	\item координата центра по $Oy$,
	\item радиус,
	\item начальный угол,
	\item конечный угол.
\end{itemize}

При этом, в AutoCAD условно принято, что дуги рисуются по часовой стрелке.

Для описания геометрии дуги в SVG используются следующие параметры:
\begin{itemize}
	\item координата начала по $Ox$,
	\item координата начала по $Oy$,
	\item радиус,
	\item флаг большой/малой дуги,
	\item флаг направления дуги,
	\item координата конца по $Ox$,
	\item координата конца по $Oy$.
\end{itemize}

Как можно заметить, для описания дуг требуется достаточно много параметров, обрабатывать которые не всегда удобно.

Поэтому AutoCAD разработали собственный формат описания геометрии дуг (внутри полилиний) в DXF. Он содержал координаты центра дуги и параметр выпуклости (bulge).

\paragraph{Параметр <<bulge>>} \label{sec:bulge}

Особый интерес представляет параметр \textit{bulge} (выпуклость), определённый в DXF-формате для каждой из вершин полилинии.
Чтобы понять сущность данного параметра, который представляет собой степень кривизны дуги окружности между двумя точками, необходимо сначала разобраться с геометрией дуг.

\begin{figure}[H]
	\centering
	\includegraphics[width=0.5\textwidth]{figures/arcgeom.png}
	\captionof{figure}{Геометрия дуги окружности}
	\label{fig:arcgeom}
\end{figure}

Так как дуга окружности описывает часть этой окружности, то она и обладает всеми атрибутами данной окружности (см. рис. \ref{fig:arcgeom}). Среди них:

\begin{itemize}
	\item Радиус ($r$) --- радиус дуги такой же, как и у окружности;
	\item Центр ($P$) --- тот же, что и у окружности;
	\item Центральный угол ($\Theta$) --- в окружности равен $360^{\circ}$;
	\item Длина дуги ($le$) --- является частью периметра (длины) окружности.
\end{itemize}

Для дальнейшей работы с геометрией дуг примем, также, следующие специфичные атрибуты:

\begin{itemize}
	\item Начальная и конечная точка ($P1, P2$) --- это «вершины» дуги. Хотя иногда и целесообразно говорить о конкретных точках, не лежащих на концах дуги;
	\item Длина хорды ($c$) --- у дуг и окружностей можно провести бесконечное количество хорд, но для нас интерес представляет только хорда, проходящая через её вершины;
	\item Середина дуги ($P3$) --- точка, делящая дуги с данными вершинами на две, равные по длине, дуги;
	\item Апофема ($a$) --- это отрезок, вершинами которого являются середина дуги и её центр. Апофема перпендикулярна хорде;
	\item Высота дуги ($s$) --- это отрезок, проведённый из середины дуги перпендикулярно к хорде.
\end{itemize}

Кроме самой себя, дуга может, также, и описывать другие геометрические формы: круговой сегмент и сектор. Обе геометрические формы включают в себя все вышеперечисленные атрибуты, однако для выведения формулы параметра \textit{bulge} (выпуклости), потребуется рассмотрение только кругового сектора.

В документации AutoCAD \cite{autocad2012dxf} выпуклостью называется тангенс четверти угла дуги между выбранной вершиной и следующей вершиной в списках вершин полилиний. Отрицательность параметра \textit{bulge} указывает на то, что дуга отрисовывается по часовой стрелке от выбранной вершины к следующей. Выпуклость, равная нулю --- прямой сегмент, выпуклость, равная единице --- половина окружности.

Проблема «расшифровки» атрибутов дуги для дальнейших манипуляций с ней заключается в том, что входными данными являются только координаты вершин и рассматриваемый параметр --- \textit{bulge}.

В самом деле, взяв арктангенс от параметра \textit{bulge} и умножив его на $4$, легко получить центральный угол, на который опирается рассматриваемая дуга. Результат получен в радианах. Для перевода значения в градусы, необходимо умножить это значение на $\pi$ и разделить на $180^{\circ}$.

Для вывода данной зависимости, рассмотрим дугу окружности (см. рис. \ref{fig:arcchord}).

\begin{figure}[H]
	\centering
	\includegraphics[width=1.0\textwidth]{figures/arcchord.png}
	\captionof{figure}{Дуга окружности с проведённой хордой и углами при ней}
	\label{fig:arcchord}
\end{figure}

Если провести к углу $\Theta$ биссектрису, то получится синий угол $\eta$. В итоге, мы получим равнобедренный треугольник (зеленый), в котором углы $\varphi$ и $\tau$ равны. Поскольку сумма углов в треугольнике всегда равна $180^{\circ}$ градусам, мы теперь знаем, что углы $\varphi$ и $\tau$ равны следующему (\ref{F:phi}):

\begin{equation}
	\varphi=\tau=\frac{(180^{\circ}-\frac{\Theta}{2})}{2}\Rightarrow\varphi=90^{\circ}-\frac{\Theta}{4}
	\label{F:phi}
\end{equation}

Теперь посмотрим на хорду $c$, проведённую от $P1$ до $P2$. Вместе с красными катетами угла $\Theta$ она тоже образует равнобедренный треугольник, а значит, $\gamma=\xi$. Угол при вершине треугольника $P-P1-P2$ --- это центральный угол $\Theta$, поэтому $\gamma$ и $\xi$ вычисляются следующим образом (\ref{F:gamma}):

\begin{equation}
	\gamma=\xi=\frac{180^{\circ}-\Theta}{2}\Rightarrow\gamma=90^{\circ}-\frac{\Theta}{2}
	\label{F:gamma}
\end{equation}

Таким образом, жёлтый угол $\varepsilon$ должен быть равняться разнице между фиолетовым углом $\varphi$ и оранжевым углом $\gamma$. Другими словами, $\varepsilon$ --- это четверть центрального угла $\Theta$ (\ref{F:epsilon}):

\begin{equation}
	\varepsilon=(90^{\circ}-\frac{\Theta}{4})-(90^{\circ}-\frac{\Theta}{2})\Rightarrow\varepsilon=\frac{\Theta}{2}-\frac{\Theta}{4}=\frac{\Theta}{4}
	\label{F:epsilon}
\end{equation}

Параметр \textit{bulge} (выпуклость) описывает, насколько дуга «выпирает» из вершин, то есть насколько велика высота дуги ($s$) (или расстояние от $P3$ до $P4$). Высота образует катет прямоугольного треугольника с углом, равным четверти центрального угла (см. жёлтый треугольник $P-P2-P3$ на рис. \ref{fig:epsilon}), и поскольку тангенс описывает отношение между катетами в прямоугольном треугольнике, легко описать геометрию с помощью этого одного угла (\ref{F:epsilon}):

\begin{equation}
	\frac{\sin(\varepsilon)}{\cos(\varepsilon)}=\tan(\varepsilon)
	\label{F:epsilon}
\end{equation}

\begin{figure}[H]
	\centering
	\includegraphics[width=0.5\textwidth]{figures/epsilon.png}
	\captionof{figure}{Связь угла $\varepsilon$ с центральным углом}
	\label{fig:epsilon}
\end{figure}

Мы, также, могли бы найти тангенс угла $\varepsilon$, просто разделив противолежащий катет на смежный катет --- что означает высоту дуги $s$, делённую на половину длины хорды $c$, --- но не зная $s$ и уже имея тангенс $\varepsilon$, полезнее найти $s$ (\ref{F:s}):

\begin{equation}
	s=\frac{c}{2}\cdot\tan(\varepsilon)
	\label{F:s}
\end{equation}

Примем

\begin{equation}
	\tan(\varepsilon)=bulge
	\label{F:tanepsilon}
\end{equation}

Тогда

\begin{equation}
	s=\frac{c}{2}\cdot bulge
	\label{F:sfinal}
\end{equation}

Таким образом, радиус дуги может быть найден следующим образом (\ref{F:r}):

\begin{equation}
	r=\frac{(\frac{c}{2})^2+s^2}{2s}
	\label{F:r}
\end{equation}

Знак той или иной выпуклости важен для определения дуги относительно вершин. Если выпуклость положительна, это означает, что дуга идёт против часовой стрелки от начальной вершины до конечной вершины. Если выпуклость отрицательна, это означает, что дуга идет, наоборот --- по часовой стрелке.

Поэтому все приведенные выше формулы должны касаться абсолютного значения выпуклости, а не фактического значения, иначе можно получить отрицательный радиус.

Итак, поняв, что $bulge = tan(\frac{\Theta}{4})$, в согласовании с документацией AutoCAD \cite{autocad2012dxf} примем, что \textit{bulge} положителен, когда при передвижении от начальной точки дуги к конечной движение происходит против часовой стрелки.

Ясно, что когда при $\Theta=0$ выпуклость $bulge(\Theta)=0$. Для углов в $180^{\circ}$: $bulge(\Theta)=\pm1$. В случае, когда $\Theta=90^{\circ}$, получим следующее (\ref{F:theta90}):

\begin{equation}
	bulge(90^{\circ})= \tan(\frac{90^{\circ}}{4})=\tan(\frac{\pi}{8})
	\label{F:theta90}
\end{equation}

Используя зависимость для тангенса половинного аргумента (\ref{F:tanhalfarg}):

\begin{equation}
	\tan(\frac{\alpha}{2})=\pm\frac{\sin(\frac{\alpha}{2})}{\cos(\frac{\alpha}{2})}=\pm\frac{2\sin^2(\frac{\alpha}{2})}{2\sin(\frac{\alpha}{2})\cos(\frac{\alpha}{2})}=\pm\frac{1-\cos(x)}{\sin(x)}
	\label{F:tanhalfarg}
\end{equation}

Для $\alpha=\frac{\pi}{8}$ получим (\ref{F:bulge90}):

\begin{equation}
	bulge(90^{\circ})=\tan(\frac{\pi}{8})=\pm\frac{1-\cos(\frac{\pi}{4})}{\sin(\frac{\pi}{4})}=\pm\frac{1-\frac{\sqrt2}{2}}{\frac{\sqrt2}{2}}=\pm\frac{1-\frac{1}{\sqrt2}}{\frac{1}{\sqrt2}}=\pm(\sqrt2-1)
	\label{F:bulge90}
\end{equation}



В результате, математические данные совпадают с документацией AutoCAD \cite{autocad2012dxf} и гласят, что

\begin{enumerate}
	\item $bulge = 0$ для отрезка прямой,
	\item $bulge = \pm1$ для дуги в $180^{\circ}$ (половина окружности),
	\item $bulge = \pm(\sqrt2-1) \approx0.41421...$ для четвертей окружностей, когда угол раствора дуги равен $90^{\circ}$.
\end{enumerate}

\section{Актуальное состояние проблемы использования файловых форматов в области САПР} \label{sec:actual}

САПР используют множество разных форматов хранения и передачи данных. Проблема заключается в унификации форматов с целью сокращения их числа и снижении нагрузки с процессов переконвертации и обработки различных форматов данных для получения необходимой информации внутри САПР.

Хотя задача полной унификации используемых форматов файлов вряд ли может быть полностью решена для современных  ПП, однако, в последние десятилетия развития технологий разработки программного обеспечения накоплены подходы, позволяющие значительно снизить остроту проблем, прежде всего за счёт продуманного использования открытых форматов хранения и обмена данными. При этом важны как функциональность формата, то есть то, какие именно данные он содержит, так и организация, то есть представление хранимых данных.

Например, для хранения и обмена геометрической информацией в САПР «Сириус» используется унаследованный двоичный формат DBS. Однако сложность чтения двоичного формата, неудобство хранения геометрической информации, а также актуальность акцент формата на экономии памяти препятствуют эффективной работе по обмену информацией между ПО. По этим причинам программистами САПР «Сириус» принято решение заменить данный формат другим --- более простым и удобным JSON.

На разных этапах как научных исследований, так и технологической подготовки производства, возникает потребность визуализации разнообразной геометрической информации, такой как геометрия деталей и ограничивающих их контуров, положение допустимых точек врезки и выключения инструмента, маршруты, получаемые в ходе решения различных классов задач резки, а также маршруты движения резака, получаемые после обработки постпроцессором и т.п.

Разработка в этих целях специальных графических утилит является традиционным подходом. Альтернативой, как пишет к.т.н. Уколов \cite{ukoloff}, является визуализация путём экспорта в удобочитаемый и поддерживаемый широкораспространёнными приложениями формат, например, в SVG. Векторные изображения, хранящиеся в этом формате, можно открывать с помощью любых современных браузеров; для формата доступно большое количество готовых библиотек; а также, формат кросс-платформенный, что означает возможность обозрения файлов данного типа на большинстве платформ и операционных систем.

В целях хранения промежуточных геометрических данных для дальнейшей обработки, а также для контроля содержания необходимых (поддерживаемых) примитивов (объектов) в DXF-файле, в рамках данной работы и проекта по разработке подсистем САПР \textit{ТокКТЭ} разработан новый формат хранения данных в текстовом документе --- TXT(DXF-type), который описан в разделе \ref{sec:aufgabe}. Также, с помощью данного формата может производится расчёт длины траектории  контура детали (обычно, в поперечном её сечении). Это может быть полезно при применении ПО в области лазерной резки с помощью станков с ЧПУ, а, в частности, в САПР <<Сириус>> \cite{petunin}.

В рамках этой работы и проекта по разработке подсистем САПР \textit{ТокКТЭ} был, также, изобретён формат хранения данных в текстовом файле --- TXT(x,y,r) в виде координат и радиуса (описание в разделе \ref{sec:aufgabe}). Он применяется для автоматизированного технологического проектирования, для формирования УП. Информация в данном формате о примитивах изображения контура детали используется для непосредственного составления УП, так как каждая последующая точка имеет не только плоские координаты, но и способ достижения этой точки (тип примитива: отрезок, если радиус равен нулю; дуга, если радиус ненулевой).

В целом, разрабатываемый набор конвертеров (модуль экспорта) представляет собой цельный ПП, сочетающий в себе набор необходимых разработчику УП начальных функций для автоматизированного технологического проектирования. Это ПО может быть интегрировано в подсистемы САПР по работе с файлами, так как по сути универсально в своём применении (используется в области 2D-резки, токарной обработке).

Несмотря на применимость в области САПР, приложение разрабатывалось с учётом возможности использования этого пользователями для их собственных целей, не связанных с САПР.


\section{Постановка задачи} \label{sec:aufgabe}

Необходимо разработать алгоритмы и приложение по конвертации данных из формата \textbf{DXF} в форматы \textbf{TXT (DXF-type)}, \textbf{TXT(x,y,r)}, \textbf{SVG}, \textbf{JSON}.

Формат TXT (DXF-type) должен содержать названия объектов (сущностей) с аттрибутом в скобках, под которыми они отображаются в DXF-формате (LINE, ARC и т.д.) в отдельных строках, после каждой из которых указываются основные параметры этих объектов. Для линий указываются координаты начала и конца. Для дуг и окружностей --- координаты центра и радиус. Для полилиний --- координаты точек, соединённых линиями. Например:
\begin{lstlisting}[label=list:dxftxtscheme]
LINE(#38)
156.732	67.105
124.332	67.105
ARC(#75)
-3.0	15.0	5.0
CIRCLE(#2B9)
0.0 0.0 20.0
LWPOLYLINE(#2A9)
0.0 0.0 10.0 0.0 10.0 10.0 0.0 10.0 
\end{lstlisting}

Формат выходного файла TXT(x,y,r) содержит в каждой строке по три параметра: координата абсциссы точки, координата ординаты и радиус перехода от данной точки к следующей (последний \textit{bulge} DXF-файла, при незамкнутом контуре полилинии, не имеет смысла). Пример содержания такого файла:
\begin{lstlisting}[label=list:dxftxtscheme]
158.33 59.61 0
108.33 59.61 0
108.33 78.61 0
118.33 68.11 0
120.33 66.11 2.00
\end{lstlisting}

Формат SVG стандартизован (см. раздел \ref{subs:svg}). При его открытии любым из доступных способов (с помощью Интернет-браузера, например) должно корректно отображаться поддерживаемые объекты входного DXF.

Формат JSON так же стандартизован (раздел \ref{subs:json}), однако после конвертации файлы в этом формате должны содержать данные по двум тэгам: \textit{partid} и \textit{paths}, например:
\begin{lstlisting}[language=json,label=list:dxftxtscheme]
[{
	"partid": "LIST",
	"paths": [
	[
	[0, 0, 0],
	[0, 2000, 0],
	[5000, 2000, 0],
	[5000, 0, 0],
	[0, 0, 0]]
	]},{
	"partid": "00112",
	"paths": [
	[
	[322, 1045, 1],
	[448, 1045, 1],
	[322, 1045, 0]],
	[
	[72, 1045, 1],
	[198, 1045, 1],
	[72, 1045, 0]],
	[
	[197, 785, 1],
	[323, 785, 1],
	[197, 785, 0]]
	]}
\end{lstlisting}

Здесь в тэге \textit{pathid} указывается наименование детали, а в тэге \textit{paths} --- примитивы, задающиеся по типу TXT(DXF-type).

Схема работы приложения представлена на рисунке \ref{fig:primiviewscheme}.

\begin{figure}[H]
	\centering
	\includegraphics[width=1.0\textwidth]{figures/primiviewscheme.png}
	\captionof{figure}{Схема работы разрабатываемого ПО}
	\label{fig:primiviewscheme}
\end{figure}

Кроме конвертации данных, необходимо предусмотреть в приложении визуализацию прочитанных из DXF-файла геометрических данных для проверки корректности чтения их программой (т.е. для верификации).

\paragraph{Главные требования ко входному DXF-файлу}

Так как данная работа нацелена на создание ПО для обработки геометрической информации 2D-объектов \textbf{специального типа}, то конвертироваться из DXF-файлов должна не вся информация, содержащаяся в них. В первую очередь, заказчиком работы было определено, что на входе будет подаваться \textbf{2D-контур} деталей типа <<Втулка>>, то есть \textbf{тел вращения}. Так как сконвертированная геометрия данных объектов в последующем предполагает разработку УП для токарных станков с ЧПУ, то геометрическая информация должна содержать определённый набор геометрических примитивов, с которым может работать система исполнительных органов станков с ЧПУ. Этот набор ограничивается тем, что исполнительные органы станков с ЧПУ способны перемещаться либо с помощью \textbf{линейной}, либо с помощью \textbf{круговой интерполяции}. Из этого следует, что для корректной работы САПР, для которых предназначаются разрабатываемые конвертеры, геометрия в DXF-файле на входе конвертеров должна состоять из, как минимум одного из представленных далее примитивов:

\begin{enumerate}
	\item линия (отрезок),
	\item полилиния,
	\item дуга,
	\item окружность.
\end{enumerate}

Остальные типы геометрии, реализуемой в формате DXF, такие как \textit{эллипс}, \textit{сплайн}, будут игнорироваться ПО.

На вход разрабатываемому конвертеру подаётся файл формата DXF. Формат DXF представляет собой совокупность данных с тегами всей информации, содержащейся в файле чертежа AutoCAD. Тегированные данные означают, что каждому элементу данных в файле предшествует целое число, называемое групповым кодом. Значение группового кода указывает, какой тип данных имеет следующий элемент. Это значение также указывает смысл элемента данных для данного типа объекта. Практически вся указанная пользователем информация в файле чертежа может быть представлена в формате DXF \cite{autocad2012dxf}.
В DXF файлах, в зависимости от их содержания, существуют сущности, представляющие для нас интерес. Среди них следующие:

\begin{enumerate}
	\item LINE (Линия),
	\item LWPOLYLINE (Полилиния),
	\item ARC (Дуга),
	\item CIRCLE (Окружность),
	\item INSERT (Вставка).
\end{enumerate}

Как уже и было отмечено, существуют и другие примитивы (ELLIPSE, SPLINE и др.), однако, основываясь на конкретных целях заказчика по возможности применения выходных файлов для генерации УП, ПП проектируется только с указанными примитивами и сущностями DXF.

Рассмотрим каждую из сущностей подробнее.

\paragraph{LINE.} Рассмотрим тэги сущности \textit{Линия}, необходимые для её реального отображения (см. табл. \ref{tab:line}).


\begin{longtable}{|l|l|}
	\caption{Рассматриваемые групповые коды сущности LINE}
	\label{tab:line}
	\centering
	\tabularnewline
	\hline
	Групповой код & Описание\\
	\hline \endfirsthead
	\subcaption{Продолжение таблицы~\ref{tab:line}}
	\\ \endhead
	\subcaption{Продолжение на след. стр.}
	\endfoot
	%\hline
	\endlastfoot
	39	&	Толщина (необязательный; по умолч. = 0)\\ \hline
	10	&	Начальная точка (в с.к. объекта) DXF: значение X\\ \hline
	20, 30	&	DXF: Y и Z значения начальной точки (в с.к. объекта)\\ \hline
	11	&	Конечная точка (в с.к. объекта)	DXF: значение X\\ \hline
	21, 31	&	DXF: Y и Z значения конечной точки (в с.к. объекта)\\ \hline
\end{longtable}

\paragraph{LWPOLYLINE.} Рассмотрим тэги сущности \textit{Полилиния}, необходимые для её реального отображения (см. табл. \ref{tab:polyline}).

\begin{longtable}{|p{70pt}|p{370pt}|}
	\caption{Рассматриваемые групповые коды сущности POLYLINE}
	\label{tab:polyline}
	\centering
	\tabularnewline
	\hline
	Групповой код & Описание\\
	\hline \endfirsthead
	\subcaption{Продолжение таблицы~\ref{tab:polyline}}
	\\ \endhead
	\subcaption{Продолжение на след. стр.}
	\endfoot
	%\hline
	\endlastfoot
	70	&	«Флаг» полилинии (бит-закодировано); по умолч. = 0; 1 – закрыта\\ \hline
	39	&	Толщина (необязательный; по умолч. = 0)\\ \hline
	10	&	Координаты вершин (в с.к. объекта), множественные вхождения; по одному вхождению для каждой вершины DXF: значение X\\ \hline
	20	&	DXF: значение Y координат вершин (в с.к. объекта), множественные вхождения; по одному вхождению для каждой вершины\\ \hline
	42	&	\textit{Bulge}. Выпуклость (множественные вхождения - для каждой вершины), (необязательно; по умолч. =0)\\ \hline	
\end{longtable}

\paragraph{ARC.} Рассмотрим тэги сущности \textit{Дуга}, необходимые для её реального отображения (см. табл. \ref{tab:arc}).

\begin{longtable}{|p{70pt}|p{370pt}|}
	\caption{Рассматриваемые групповые коды сущности ARC}
	\label{tab:arc}
	\centering
	\tabularnewline
	\hline
	Групповой код & Описание\\
	\hline \endfirsthead
	\subcaption{Продолжение таблицы~\ref{tab:arc}}
	\\ \endhead
	\subcaption{Продолжение на след. стр.}
	\endfoot
	%\hline
	\endlastfoot
	39	&	Толщина (необязательный; по умолч. = 0)\\ \hline	
	10	&	Центр дуги (в с.к. объекта)
	DXF: значение X\\ \hline	
	20, 30	&	DXF: Y и Z значения центра дуги (в с.к. объекта)\\ \hline	
	40	&	Радиус\\ \hline	
	50	&	Начальный угол\\ \hline	
	51	&	Конечный угол\\ \hline	
\end{longtable}

\paragraph{CIRCLE.} Рассмотрим тэги сущности \textit{Окружность}, необходимые для её реального отображения (см. табл. \ref{tab:circle}).

\begin{longtable}{|p{70pt}|p{370pt}|}
	\caption{Рассматриваемые групповые коды сущности CIRCLE}
	\label{tab:circle}
	\centering
	\tabularnewline
	\hline
	Групповой код & Описание\\
	\hline \endfirsthead
	\subcaption{Продолжение таблицы~\ref{tab:circle}}
	\\ \endhead
	\subcaption{Продолжение на след. стр.}
	\endfoot
	%\hline
	\endlastfoot
	39	&	Толщина (необязательный; по умолч. = 0)\\ \hline	
	10	&	Центр дуги (в с.к. объекта)
	DXF: значение X\\ \hline	
	20, 30	&	DXF: Y и Z значения центра дуги (в с.к. объекта)\\ \hline	
	40	&	Радиус\\ \hline	
	50	&	Начальный угол\\ \hline	
	51	&	Конечный угол\\ \hline	
\end{longtable}

\paragraph{INSERT.} Данная сущность представляет собой вставку блоков с геометрией. Её необходимо рассматривать, так как геометрия может быть вложенной и, таким образом, не видна обзорщиком сущностей, так как вложена. У этой сущности поиск информации по тэгам в программе не потребуются.


\section{Выводы по главе \ref{cha:aktuell}}
Анализ состояние вопросов конвертации данных и графических файлов DXF показал, что:
\begin{itemize}
	\item Задача по конвертации данных из разных форматов с целью обмена графической информацией в САПР сложна и требует, как специальных знаний в области изучаемых форматов, так и опыт в разработке алгоритмов и ПО по ним для САПР;
	\item Существующие предложения открытого рынка по ПП, конвертирующим DXF файлы. Однако эти решения не удовлетворяют потребностям, предъявляемым к выходным форматам данных;
	\item Для корректной работы программы, разработчику алгоритмов и ПО необходимы знания геометрии примитивов (в особенности, дуг), с которыми ведётся работа.	
\end{itemize}
