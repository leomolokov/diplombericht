\Defines % Необходимые определения. Вряд ли понадобться

В настоящей работе применяются следующие термины с соответствующими определениями:

\begin{description}

\item[Программный продукт] объект, состоящий из программ, процедур, правил, а также, если предусмотрено, сопутствующих им документации и данных, относящихся к функционированию системы обработки информации. ГОСТ 28806-90 «Качество программных средств. Термины и определения» (утверждён и введён в действие Постановлением Госстандарта СССР от 25 декабря 1990 г. № 3278).

\item[Система автоматизированного проектирования] организационно-техническая система, входящая в структуру проектной организации и осуществляющая проектирования при помощи комплекса средств автоматизированного проектирования (КСАП). ГОСТ 23501.101-87 «Системы автоматизированного проектирования. Основные положения» (утверждён и введён в действие Постановлением Государственного комитета СССР по стандартам от 26.06.87 № 2668).

\item[Язык разметки] набор символов или последовательностей символов, вставляемых в текст для передачи информации о его отображении или строении.

\item[Конвертация данных] преобразование данных из одного формата в другой с сохранением основного логически-структурно содержания информации.

\end{description}

%%% Local Variables:
%%% mode: latex
%%% TeX-master: "rpz"
%%% End:
