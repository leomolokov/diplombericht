\Introduction

\paragraph{Актуальность темы исследования.} В настоящее время в рамках проектов Уральского Федерального Университета (УрФУ) и Уральского межрегионального научно-образовательного центра (УМНОЦ) происходит разработка специализированных САПР. Создаваемые ПП работают с файлами, содержащими геометрическую 2D-информацию. Используются следующие форматы файлов:

\begin{enumerate}
	\item DXF,
	\item TXT,
	\item SVG,
	\item JSON.
\end{enumerate}

В связи с этим требуется разработать ПО по конвертации данных форматов файлов.

\paragraph{Степень разработанности темы исследования.} В открытом доступе сети Интернет существуют онлайн-конвертеры файлов DXF в другие форматы. Большая их часть лишь визуализирует графическую информацию, представленную в том или ином DXF-файле, и конвертирует в наиболее популярные форматы изображений (JPEG, PNG). Для использования конвертера на предприятии-заказчике необходим особый формат данных, в которые конвертируется DXF. Такие форматы не могут быть обеспечены существующим в открытом рынке ПО --- таким, как, например, <<DXF Reader GT>> от компании «Gray Technical». Несмотря на высокую степень проработки программы, формат выходных данных не соответствует требованиям, предъявляемым к TXT-, SVG- и JSON-файлам, компании <<Униматик>>. Ещё одним фактором, препятствующим компанией <<Униматик>> использование зарубежного ПО, является ежемесячная плата разработчикам. Одной из задач современной Российской Федерации является импортозамещение в машиностроительной сфере, поэтому разработка собственного ПП увеличит независимость и самостоятельность предприятий и компаний, пользующихся этим ПП.

\paragraph{Цель работы} заключается в разработке алгоритмов и программного обеспечения для чтения данных из файлов типа DXF с геометрической информацией объектов специального типа, извлечении из них информации о геометрических примитивах, вывода полученной графической информации на экран для верификации, формирования двух видов текстовых файлов (TXT(DXF-type), TXT(x,y,r)) с выводом данных в исходном виде DXF, формирования текстового файла (TXT) с выводом данных в виде координат точек и радиуса примитивов (линий, дуг) между ними, формировании файла-описания двумерной векторной графики в формате SVG с выводом данных в исходном виде DXF, формировании текстового файла формата JSON с выводом данных в виде координат точек и степенями кривизны примитивов между ними.

Данные конвертеры необходимы, в частности, при разработке таких САПР, как Сириус, ТокКТЭ и других.

Для достижения этой цели поставлены следующие \textbf{задачи}:

\begin{enumerate}[1)]
	\item выявить структурно-содержательные особенности файлов форматов DXF, TXT(DXF-type), TXT(x,y,r), SVG, JSON для последующей работы с разрабатываемым ПО;
	\item проанализировать возможность применения разных ЯП для разработки ПО;
	\item разработать алгоритмы;
	\item разработать ПО;
	\item провести анализ экономической целесообразности разрабатываемого проекта.
\end{enumerate}

\textbf{Объект исследования} --- форматы файлов по хранения и обмену графической и другой информацией компьютерных чертежей.

\textbf{Предмет исследования} --- проектирование алгоритмов и ПО для конвертации файлов из формата DXF в форматы TXT, SVG, JSON.

\paragraph{Теоретическая и практическая значимость работы}

\begin{enumerate}
	\item разработка универсальна, так как данные файловые конвертеры могут быть использованы любым пользователем в собственных целях (и вне САПР);
	\item подход к разработке конвертеров универсален, что значит, что алгоритмы и методы, применённые в данной работе, могут быть использованы при создании аналогичного ПО;
	\item разработанное ПО можно внедрять в существующие подсистемы САПР, работающие с форматом файлов DXF;
	\item САПР, использующие разработанный модуль по конвертации файлов с графической информацией, получают независимость от использования аналогичного ПО зарубежного производства;
	\item положенные в основу алгоритмы и написанное ПО имеют открытый доступ в сети Интернет и имеют перспективу развития в полноценный модуль импорта/экспорта файлов;
	\item коммерческая выгода заказчика.
\end{enumerate}