\chapter{Экономическая часть}
\label{cha:economy}

В данном разделе описаны экономические аспекты проекта по созданию ПО <<Primiview>> для обработки геометрической информации 2D-объектов специального типа.

\section{Экономическое обоснование}

Целью экономического обоснования проекта является представление разработанного ПП в качестве проекта для реализации на предприятии, что позволит провести планирование и корректировку последовательности работ (при необходимости).

\subsection{Разработка проекта}

\paragraph{Цель проекта} --- создание модуля конвертации форматов, как отдельного ПП, для внедрения в ПО для автоматизации технологического проектирования обработки деталей типа <<Втулка>> на станках с ЧПУ к 05.06.2023.

\textit{Обоснование цели}. Клиент --- это производитель станков с ЧПУ. Для создания УП на продаваемые станках клиенты заказчика привлекают САПР, разработанные за рубежом, что негативно влияет на независимость от сторонних организаций, ограниченность в дополнении ПО функционалом, необходимым только на отдельном предприятии, а также, на сохранность информации, используемой при разработке УП. Компания работает над проектом разработки своей CAM-системы для предложения своим клиентам специализированного ПО вместе с оборудованием с целью решить вышеуказанные проблемы. 

Численные критерии сравнения состояний системы клиента:
\begin{itemize}
	\item сложность проектирования,
	\item сроки (время) проектирования,
	\item себестоимость проектирования,
	\item качество результатов проектирования.
\end{itemize}

\textit{Пояснения к критериям}. Под сложностью проектирования понимается минимально-необходимая квалификация проектировщика для выполнения задач технологического проектирования.

Трудоёмкость проектирования, в данном случае, выражается в среднем времени создания (написания) одной УП.

Себестоимость проектирования оценивается не в расчёте на одну УП, а в рамках одного рабочего года. В себестоимость проектирования входят такие элементы, как зарплата сотрудника, производящего проектирование, а также, цена годовой лицензии/контракта обслуживания CAM-системы для данного количества оборудования на предприятии.

В качестве численного критерия для оценки качества результатов проектирования принято среднее количество типов ошибок, корректировки по которым оператор станка с ЧПУ вносит после того, как автоматизированное проектирование выполнено.

\textbf{Текущее состояние системы клиента}: основное ПО по автоматизации технологического проектирования введено в эксплуатацию, базовые потребности системы по конвертации DXF-файлов работают.

По численным критериям:
\begin{itemize}
	\item сложность проектирования: инженер-технолог II категории по квалификационному справочнику \cite{qualification},
	\item сроки (время) проектирования: 3 часа,
	\item себестоимость проектирования: 1420000 руб. (см. раздел \ref{par:ecocalc}),
	\item качество результатов проектирования: 10 ошибок.
\end{itemize}

\textbf{Целевое состояние системы клиента}: усовершенствованная, более гибкая и универсальная версия этого ПО.

По численным критериям:
\begin{itemize}
	\item сложность проектирования: инженер-технолог (без категории) по квалификационному справочнику \cite{qualification},
	\item сроки (время) проектирования: максимум 1,5 часа,
	\item себестоимость проектирования: максимум 1 млн.руб.,
	\item качество результатов проектирования: максимум 5 типов ошибок.
\end{itemize}

\textbf{Результатом} проекта является созданные и готовый к работе ПП, на вход которому подаётся DXF и информация о поддерживаемых примитивов в котором конвертируется в файлы форматов TXT, SVG и JSON. Файлы последних форматов содержат в себе геометрическое описание объектов по правилам, описанным в предыдущих разделах ВКР, из входного DXF, удобное для дальнейшей работы в CAM-системе "ТокКТЭ".

Команда проекта сформирована из 3 человек, среди которых:
\begin{enumerate}
	\item 1 владелец,
	\item 2 программиста, среди которых:
	\begin{enumerate}
		\item[б.1)] 1 разработчик,
		\item[б.2)] 1 тестировщик.
	\end{enumerate}
\end{enumerate}

Владелец проекта организует работу остальной команды, проводит планирование проекта, оценку его экономической эффективности, контроль за выполнением подчинёнными задач проекта.

Программисты занимаются непосредственно созданием продукта проекта, то есть написанием ПО. Разработчики отвечают за написание программного кода по техническому заданию проекта. Тестировщики выполняют проверку работоспособности ПП, ищут и сообщают отделу разработчиков о найденных и необходимых к устранению ошибок и недочётов программы.




\subsection{Дерево задач проекта}

Целью данного этапа является построение иерархического дерева, включающего в себя последовательное разбиение общей цели проекта на подцели и задачи.

\paragraph{Первый уровень иерархии.}

Главной целью проекта, как уже было сказано, является разработка программного обеспечения <<Primiview>> по конвертации файлов в формате DXF в форматы TXT, SVG, JSON. Данная цель в проекте единственная и находится на высшем уровне иерархии.

\paragraph{Второй уровень иерархии.}

В целях определения и формализации цели, структуры и методов проекта, чтобы исключить неоднозначное их понимание и толкование исполнителями, первый этап, стоящий в иерархии на втором уровне, --- это \textit{формирование технического задания (ТЗ)}.

При параллельном методе разработке, когда этапы проекта могут начинаться тогда, пока предыдущие ещё не закончились, следующим шагом данного уровня будет \textit{разработка алгоритмов} программного обеспечения. Данный этап необходим, так как именно на нём ещё можно решить несостыковки в логической части программы, исправление которых на следующих этапах редко бывает возможным.

Параллельно с разработкой алгоритмов начинается этап непосредственно \textit{разработки ПО}. Эти этапе происходят одновременно, так как они тесно взаимосвязаны, и, например, не зная на каком ЯП будет разрабатываться ПО, трудно будет рационально подобрать алгоритмы, отвечающие возможностям ЯП.

Завершающим этапом второго уровня является \textit{сдача проекта} заказчику. Эта стадия может быть выполнена только при полном выполнении предыдущих стадий, если иное не было предварительно оговорено с заказчиком.

\paragraph{Третий уровень иерархии.}

Этап формирования ТЗ подразделяется на следующие задачи:

\begin{enumerate}
	\item Определение назначения ПО. Здесь формализуются цели и функции ПП. Впоследующем, они вносятся в ТЗ. Это смысл выполнения проекта, то, к чему стремится вся его команда, что хочет получить в итоге заказчик;
	\item Исследование степени разработанности. На стадии предпроектного исследования выполняется проверка, существуют ли аналоги данного продукта в открытом доступе рынка. Если есть, то чем заказчика не устраивает их использование;
	\item Требования к продукту. Определяются численные критерии, которым должен соответствовать результат проекта на этапе его сдачи;
	\item Определение сроков. Всем участникам проекта необходимо знать, к какому сроку они должны выполнить определённый ранее объём работ. Сотрудникам, при согласовании ТЗ, необходимо оценить эти сроки, и в случае невозможности из соблюдения, просить корректировки и согласования с заказчиком более позднего выполнения задач;
	\item Написание ТЗ. Завершающая стадия этапа формирования ТЗ --- здесь собирается вся информация с предыдущих стадий и документируется согласно стандартам организации. ТЗ должно быть согласовано со всему участниками проекта.
\end{enumerate}

Этап разработки алгоритмов подразделяется на следующие стадии:
\begin{enumerate}
	\item Определение принципа работы ПО. Составляется принципиальная схема работы ПО, связь модулей, определяется предназначение каждого из модулей;
	\item Разработка алгоритмов для модулей ПП. Происходит решение поставленных задач на уровне логики и математики, строится набор взаимосвязей алгоритмов. По возможности, рассматривается применение уже существующих универсальных (реже, специальный) алгоритмов.
\end{enumerate}

Этап разработки ПО содержит такие стадии:
\begin{enumerate}
	\item Выбор ЯП. Данная стадия подразумевает проведение сравнительного анализа существующих инструментов различных ЯП применительно к разрабатываемому ПП;
	\item Определение структуры ПО. Эта стадия необходима для понимания разработчика функционала каждого из модулей программы. От этого зависит, на какие части (из каких файлов) будет состоять ПО;
	\item Написание и отладка программного кода. Главная часть создания продукта. На этой стадии разработчики непосредственно пишут программный код и отлаживают его работу.
	\item Тестирование ПО. Команда тестировщиков, также, пишет тестовый код, который проверяет разрабатываемое ПО на корректность работы его функционала.
\end{enumerate}

Завершающий этап на втором уровне иерархии --- сдача ПО, содержит следующие стадии:
\begin{enumerate}
	\item Презентация. Команда проекта презентует результаты своей работы заказчику, демонстрирует работу программы. Отчитывается по выполнению всех этапов, указанных в ТЗ;
	\item Внесение правок. По итогам презентации команда проекта заслушивает обратную связь от заказчика и на этой основе вносит в проект правки;
	\item Передача ПО заказчику. Организатор проекта передаёт заказчику исходный код ПО, документацию и инструкции, требования, прочие исходные файлы, информацию, обеспечивающую доступ к ПП;
	\item Обучение Сотрудников. При сдаче нового ПП, команде разработчиков необходимо обучить персонал заказчика работе с новым ПО;
	\item Устранение недостатков. После обучения исполнитель проекта (организация) предоставляет заказчику 7 рабочих дней на проведение внутренних проверок ПО, по результатам которых заказчик предоставляет исполнителю список исправлений, которые команда проекта обязана скорректировать в разработанном ПО;
	\item Закрытие задания на проект. Расчёт. Конечным мероприятием сдачи ПО являет закрытие сторонами задания по договору и получение расчёта за выполненную работую.
\end{enumerate}

\paragraph{Четвёртый уровень иерархии.}

Описание данного уровня иерархии приведено кратко для примера. В самом деле, каждая из стадий третьего уровня подразделяется на задачи четвёртого уровня.

Стадия исследования степени разработанности проблемы в этапе формирования ТЗ подразделяется на следующие задачи:
\begin{enumerate}
	\item Исследование отечественного рынка аналогичных ПП. Задача состоит в поиске решений по аналогичным проектам в открытом доступе в рамках отчественного рынка;
	\item Исследование зарубежного рынка аналогичных ПП. Данная задача отличается от предыдущей сложностью поиска аналогов (на зарубежном рынке(пространстве)), так как для проведения данного анализа необходим высокий уровень владения иностранным (английским) языком, а также, требуется знание достоверных ресурсов (источников) информации.
\end{enumerate}

Описанные элементы иерархии сведены в иерархическое дерево (см. рисунок \ref{fig:projektaufgaben}). Уровни иерархической структуры оформлены таким образом, что, чем конкретнее описаны элементы структуры, тем насыщеннее цвет заливки.

	
\begin{figure}[H]
	\centering
	\includegraphics[width=1.0\textwidth]{figures/projektaufgaben.png}
	\captionof{figure}{Дерево задач проекта}
	\label{fig:projektaufgaben}
\end{figure}

\subsection{Построение диаграмм проекта}

На данном этапе строится диаграмма Ганта (англ. Gantt Chart, также, ленточная диаграмма, график Ганта) --- это тип столбчатых диаграмм, использующийся для иллюстрации плана, графика работ проекта. Также, является методом планирования проекта. Эта диаграмма представляет собой отрезки (графические плашки), размещающиеся на горизонтальной шкале времени. Каждый отрезок соответствует отдельно задаче (подзадаче). Начало, конец и длина отрезка на шкале времени соответствуют началу, концу и длительности задачи. Диаграмма может использоваться для представления текущего состояния выполнения работ: часть прямоугольника, отвечающего задаче, заштриховывается, отмечая процент выполнения задачи; показывается вертикальная линия, отвечающая моменту «сегодня».

Рядом с самой диаграммой располагается таблица со списком работ, строки которой соответствуют отдельным задачам, отображённым на диаграмме, в то время как столбцы содержат дополнительную информацию о задаче.


%\begin{sidewaysfigure}[H]
%	\centering
%	\includegraphics[width=\textwidth, height=0.5\textheight]{figures/gantt.png}
%	\captionof{figure}{Диаграмма Ганта для проекта}
%	\label{fig:gantt}
%\end{sidewaysfigure}


%\begin{landscape}
%\begin{sideways}
%\KOMAoptions{paper=landscape, pagesize}
%\recalctypearea
%\begin{landscape}
\begin{figure}[H]
	\centering
	\includegraphics[width=1.4\textwidth, angle=90]{figures/gantt.png}
	\captionof{figure}{Диаграмма Ганта для проекта}
	\label{fig:gantt}
\end{figure}
%\end{landscape}
%\KOMAoptions{paper=portrait, pagesize}
%\recalctypearea
%\end{sideways}
%\end{landscape}


















\section{Сравнительная экономическая эффективность}

Расчеты сравнительной экономической эффективности капитальных вложений (инвестиций) применяются при сопоставлении нескольких возможных для осуществления вариантов инженерных решений: при решении задач по выбору взаимозаменяемых материалов, внедрению новых видов техники, модернизации оборудования, способов организации производственных процессов и т. п. То есть для оценки решений, которые являются альтернативными для обеспечения одинаковых конечных результатов деятельности. При этом конечные результаты (производство конкретной продукции с определенными характеристиками в заданном объеме) уже известны, есть необходимость определить, какой способ ее изготовления на том или ином этапе деятельности предприятия является более выгодным.

\subsection{Исходные данные}

Станкостроительное предприятие рассматривает заказ на создание программного обеспечения для своего оборудования (токарных станков с ЧПУ). Это ПО автоматизирует процесс создания управляющих программ для станков с ЧПУ, взамен работе инженера-технолога-программиста, который, обычно, берёт чертёж детали и либо вручную пишет УП, либо использует иностранные CAM-системы, предварительно создавая 3d-модель по выданному чертежу.
В рамках данной (третьей) части ВКР будет рассматриваться инвестиционный проект (ИП) с точки зрения покупателя оборудования у предприятия, которое привлекло силы университета для создания описанного ПО. Сравниваются два варианта – покупка станков без ПО и, соответственно, с ним.

\paragraph{Сравнительная характеристика вариантов.} Рассмотрим ситуацию с точки зрения покупателя оборудования рассматриваемого станкостроительного предприятия. Соберём основные данные в таблицу  \ref{tab:startcomparis}.

\begin{longtable}{|p{0.50\textwidth}|p{0.50\textwidth}|}%{|p{0.40\textwidth}|c|}
	\caption{Сравнительная характеристика вариантов ИП}
	\label{tab:startcomparis}
	\centering
	\tabularnewline
	\hline
	Вариант 1      & Вариант 2\\
	\hline \endfirsthead
	\subcaption{Продолжение таблицы~\ref{tab:startcomparis}}\\
	\hline \endhead
	\subcaption{Продолжение на след. стр.}
	\endfoot
	%\hline
	\endlastfoot
	Покупка станка без ПО	&	Покупка станка вместе с ПО (CAM-системой) за большую цену\\
	\hline
	Наём во время этапа подготовки производства инженеров-технологов-программистов для написания УП	&	Привлекаются технологи из имеющегося штата сотрудников для выполнения дополнительных обязанностей по контролю создания УП с помощью купленного ПО\\
	\hline
	Время написания УП в три раза дольше, чем во втором варианте	&	Время написания УП в течение одного часа (в среднем)\\
	\hline
\end{longtable}

Варианты рассматриваются с точки зрения потребителя оборудования.
За сопоставляемые характеристики принимаются следующие:

\begin{itemize}
	\item Объём производства (серийное),
	\item Частота создания УП в год (200 новых УП в среднем).
\end{itemize}

\paragraph{Выбор единичного периода времени.} В качестве единичного периода времени для расчётов примем один год, так как на рассматриваемом предприятии-клиенте ситуация с производством каждый месяц практически не меняется. Также, большинство справочных величин ссылаются именно на годовой период, что тоже является подтверждением равномерной распределённости экономических характеристик внутри отдельно взятых месяцев.

\paragraph{Состав и описание капитальных вложений по вариантам.} В капитальные вложения входят следующие величины:

\begin{itemize}
	\item Цена станка с без ПО – 2 750 000 руб.,
	\item Цена встроенной CAM-системы на единицу оборудования– 70 000 руб.,
	\item Наладка полной группы станков – 20 000 руб.
\end{itemize}

\paragraph{Принятие решения по нормативному сроку окупаемости и его обоснование.} Соответствует требованиям к сроку окупаемости дополнительных капитальных вложений, в данном случае – в токарный станок с ЧПУ.

Срок полезного использования оборудования – 10 лет.

Срок контракта на выпуск продукции с использованием данного оборудования – в рассматриваемой ситуации нет ограничений, токарная обработка постоянно проводится на предприятии.

Требования собственника, инвестора – предприятие установило желаемый срок окупаемости – 5 лет.

Следовательно, задаём Тн (нормативный срок окупаемости) равным 5 лет, так как временные рамки требований инвестора меньше срока полезного использования оборудования.

\paragraph{Определение состава затрат по вариантам (результат – перечень затрат).} Корректировка затрат в соответствии с возможностями Методики сравнительной эффективности (включаем в расчет только различающиеся по альтернативам затраты). Деление затрат на переменные и постоянные. Формирование списка исходных данных для выполнения расчетов (см. таблицу \ref{tab:gegeben}).

\begin{longtable}{|p{0.30\textwidth}|p{0.30\textwidth}|p{0.30\textwidth}|}
	\caption{Исходные данные для расчётов текущих затрат}
	\label{tab:gegeben}
	\centering
	\tabularnewline
	\hline
 	\quad & \multicolumn{1}{c|}{Вариант 1} & \multicolumn{1}{c|}{Вариант 2}\\
	\hline \endfirsthead
	\subcaption{Продолжение таблицы~\ref{tab:gegeben}}
	\\ \hline \endhead
	\subcaption{Продолжение на след. стр.}
	\endfoot
	%\hline
	\endlastfoot
	\multicolumn{3}{|c|}{Переменные затраты (на единицу объема деятельности (одну УП))}\\
	\hline
	Зарплата технолога, руб & \multicolumn{1}{c|}{1500} & \multicolumn{1}{c|}{1500}\\
	\hline
	Время на написание одной УП, час & \multicolumn{1}{c|}{3} & \multicolumn{1}{c|}{1}\\
	\hline
	\multicolumn{3}{|c|}{Постоянные затраты (на единицу оборудования)}\\
	\hline
	Цена годовой лицензии/контракта обслуживания CAM-системы, руб & \multicolumn{1}{c|}{50000} & \multicolumn{1}{c|}{55000}\\
	\hline
\end{longtable}

\subsection{Расчёты и анализ}

Так как выбран нормативный срок окупаемости, равный одному году, то к нему будут приведены расчёты по приведённым затратам.

\paragraph{Исходные данные.}

Среднее годовое количество УП на предприятии-покупателе станков
\[N=200 \text{ шт;}\]

Срок полезного использования оборудования:
\[T_{machinery}=10 \text{ лет;}\]

Требования инвестора по окупаемости ИП:
\[T_{inv}=5 \text{ лет;}\]

Принятая норма окупаемости:
\[{T_n=T}_{inv}=5 \text{ лет;}\]

Наладка полной группы станков:
\[{CAM}_{Term}=20000 \text{ руб;}\]

Цена встроенной CAM-системы на единицу оборудования:
\[{CAM}_2=70000 \text{ руб;}\]

Цена станка без встроенной CAM-системы:
\[M=2750000 \text{ руб;}\]

Цена годовой лицензии/контракта обслуживания CAM-системы на единицу оборудования для вариантов 1 и 2, соответственно:
\[{CAM}_{1Perm}=50000 \text{ руб;}\]

\[{CAM}_{2Perm}=55000 \text{ руб;}\]

Среднее время написания одной УП:
\[t_1=3 \text{ часа;}\]
\[t_2=1 \text{ час;}\]

Почасовая оплата технолога:
\[Sal=1500 \text{ руб;}\]

Страховые сбора от заработной платы:
\[fees=30\%\]

Количество покупаемых станков:
\[N_M=5 \text{ шт;}\]

\paragraph{Расчёт.}\label{par:ecocalc}

Себестоимость использования оборудования и ПО:
\begin{equation}
	\begin{aligned}
		C_1 = Sal \cdot t_1\cdot N \cdot(100\%+fees)+N_M \cdot CAM_{1Perm} =\\= 1500 \cdot 3 \cdot 200 \cdot (100\%+30\%)+5 \cdot 50000=1 420 000 \text{ руб;}
	\end{aligned}
\end{equation}
\begin{equation}
	\begin{aligned}
		C_2 = Sal \cdot t_2\cdot N \cdot(100\%+fees)+N_M \cdot CAM_{2Perm} =\\= 1500 \cdot 1 \cdot 200 \cdot (100\%+30\%)+5 \cdot 55000=665 000 \text{ руб;}
	\end{aligned}
\end{equation}

Условно-годовая экономия (на себестоимости):
\begin{equation}
	\begin{aligned}
		E=|C_1-C_2|=|1420000-665000|=755 000 \text{ руб;}
	\end{aligned}
\end{equation}

Капитальные вложения предприятия-покупателя станков:
\begin{equation}
	\begin{aligned}
		K_1=M \cdot N_M+{CAM}_{Term}=2750000 \cdot 5+20000=13 770 000 \text{ руб;}
	\end{aligned}
\end{equation}
\begin{equation}
	\begin{aligned}
		K_2=N_M \cdot (M+{CAM}_2)+{CAM}_{Term}=\\=5 \cdot (2750000+70000)+20000 = 14 120 000 \text{ руб;}
	\end{aligned}
\end{equation}

Дополнительные капитальные вложения:
\begin{equation}
	\begin{aligned}
		K_{extr}=|K_1-K_2|=|13770000-14120000|=350 000 \text{ руб;}
	\end{aligned}
\end{equation}
\begin{equation}
	\begin{aligned}
		K_{extr}=N_M \cdot {CAM}_2=5 \cdot 70000=350 000 \text{ руб. (проверка);}
	\end{aligned}
\end{equation}

Срок окупаемости дополнительных капитальных вложений:
\begin{equation}
	\begin{aligned}
		T_{payback}=\frac{K_{extr}}{E}=\frac{350000}{755000}=0,464 \text{ лет;}
	\end{aligned}
\end{equation}

Приведённые затраты по вариантам:
\begin{equation}
	\begin{aligned}
		Z_1=C_1+\frac{1}{T_n} \cdot K_1=1420000+\frac{1}{5} \cdot 13770000=4 174 000 \text{ руб;}
	\end{aligned}
	\label{F:spends1}
\end{equation}
\begin{equation}
	\begin{aligned}
		Z_2=C_2+\frac{1}{T_n} \cdot K_2=665000+\frac{1}{5} \cdot 14120000=3489000 \text{ руб;}
	\end{aligned}
	\label{F:spends2}
\end{equation}

Годовой экономический эффект:
\begin{equation}
	\begin{aligned}
		E_{annual}=|Z_1-Z_2|=|4174000-34890000|=685000 \text{ руб;}
	\end{aligned}
\end{equation}

Минимальный годовой объём деятельности, при котором обеспечивается приведённый годовой экономический эффект:
\begin{equation}
	\begin{aligned}
		N_{cr}=\frac{N_M \cdot CAM_{1Perm}-N_M \cdot CAM_{2Perm}-\frac{K_{extr}}{T_n}}{S \cdot t_2 \cdot (100\%+fees)-Sal \cdot t_1 \cdot (100\%+fees)}=\\=\frac{5 \cdot 50000-5 \cdot 55000-\frac{350000}{5}}{1500 \cdot 1 \cdot 100\%+30\%-1500 \cdot 3 \cdot 100\%+30\%}=24,359 \text{ шт;}
	\end{aligned}
\end{equation}

По вычисленным в формулах \ref{F:spends1}, \ref{F:spends2} затратам изобразим на графике (см. рисунок \ref{fig:economyborders}) границы целесообразности рассматриваемых вариантов.

\pgfplotsset{width=13cm,compat=1.9}
\begin{figure}[H]
	\centering
	\begin{tikzpicture}
		\centering
		\begin{axis}[ 
			xlabel = {$N, \text{шт. (объём деятельности)}$},
			ylabel = {$Z, \text{руб. (приведённые затраты)}$},
			xmin=0, xmax=200, ymin=2900000,
			xtick={0,20,40,60,80,100,120,140,160,180,200},
			ymajorgrids=true,
			xmajorgrids=true
			]
			\addplot[line width=1mm,  blue, mark=*] coordinates {
				(0,3004000) (200,4174000)
			};
			\addplot[line width=1mm,  red, mark=triangle] coordinates {
				(0,3099000) (200,3489000)
			};
%			\node at (axis cs:0,4200000) [anchor = north west] {\text{Границы целесообразности}};
		\end{axis}
		\label{graph:economyborders}
	\end{tikzpicture}
	\captionof{figure}{Границы целесообразности рассматриваемых вариантов}
	\label{fig:economyborders}
\end{figure}

Получив необходимые значения по критериям сравнения, сведём результаты в таблицу \ref{tab:fintabeco}).

%\begin{longtable}{|p{0.40\textwidth}|p{0.05\textwidth}|p{0.15\textwidth}|p{0.15\textwidth}|p{0.15\textwidth}|}
\begin{longtable}{|p{0.40\textwidth}|p{0.05\textwidth}|p{0.13\textwidth}|p{0.13\textwidth}|p{0.13\textwidth}|}
	\caption{Сравнительная характеристика рассматриваемых вариантов по показателям эффективности}
	\label{tab:fintabeco}
	\centering
	\tabularnewline
	\hline
	\multicolumn{1}{|c|}{\multirow{3}{6cm}{Наименование показателя}} & \multicolumn{1}{c|}{\multirow{3}{1cm}{Ед. изм.}} & \multicolumn{2}{c|}{По вариантам:} & \multicolumn{1}{c|}{\multirow{3}{3cm}{Отклонения показателей}}\\
	\cline{3-4} 
	\endfirsthead
	\subcaption{Продолжение таблицы~\ref{tab:fintabeco}}
	\\ \hline \endhead
	\subcaption{Продолжение на след. стр.}
	\endfoot
	%\hline
	\endlastfoot
	\multicolumn{1}{|c|}{}&\multicolumn{1}{c|}{}&Вариант без ПО&Вариант c ПО&\multicolumn{1}{c|}{}\\
	\hline
	Годовой объем деятельности&шт.&200&200&-\\
	\hline
	Капитальные вложения, всего&руб.&13770000&14120000&350000\\
	\hline
	\multicolumn{5}{|l|}{в том числе:}\\
	\hline
	Наладка станков&руб.&20000&20000&-\\
	\hline
	Цена станка (Вар. 2 + ПО)&руб.&13750000&665000&755000\\
	\hline
	Срок окупаемости дополнительных кап. вложений&&0,464&&\\
	\hline
	Приведённые затраты по вариантам&руб.&4174000&3489000&658000\\
	\hline
	Годовой экономический эффект&&&685000&\\
	\hline
\end{longtable}


\subsection{Выводы по результатам расчётов.}

Так как на первых этапах расчёта по методу сравнительной эффективности ИП нельзя было сделать конкретный вывод по поводу целесообразности одного из предлагаемых вариантов по причине того, что по первому варианту себестоимость ИП была больше в сравнении со вторым, а капитальные вложения, соответственно, меньше, то расчёт был продолжен до момента вычисления расчётного срока окупаемости дополнительных капитальных вложений, а также расчёта приведённых затрат по каждому из вариантов.

Исходя из расчётов и построенного по ним графика, сделаем вывод, что, производя уже 25 УП за год, выгоднее становится вариант с ПО, так как приведённые затраты для соответствующего количество производимых УП для этого варианта оказываются меньше.

Анализируя итоговые данные, выбираем для реализации второй вариант, то есть покупка оборудования вместе со встроенным ПО (CAM-системой), объясняя выбор тем, что расчётный срок окупаемости оказался намного меньше рассматриваемого нормативного срока окупаемости (0,4 и 5 лет, соответственно), а приведённые затраты по первому варианту оказались больше, чем по второму.

Действительно, экономия времени на создании УП нивелирует большие капитальные вложения на этапе инвестиционного периода УП.


%%% Local Variables:
%%% mode: latex
%%% TeX-master: "rpz"
%%% End:
