% Также можно использовать \Referat, как в оригинале
\begin{abstract}
	
	Отчёт содержит \formbytotal{page}{страниц}{у}{ы}{}, 23 рисунка, 8 таблиц, 40 формул, 1 приложение.
	
%	Отчёт содержит \formbytotal{page}{страниц}{у}{ы}{}, \formbytotal{totalcount@figure}{рисун}{ок}{ка}{ков}, \formbytotal{totalcount@table}{таблиц}{у}{ы}{}, \ref{LastNumberedEquation} формул, \formbytotal{totalappendix}{приложен}{ия}{ий}{}.
	
	%\totfig \tottab
	
	%g2-105 package - why doesnt work!?!?!?
	%\totalpages
	%
	%\totaltables
	%\totalfigures
	%\totalbibs  
	%\pageref{reftotalpage}
	
	Ключевые слова: конвертеры, алгоритмы, программное обеспечение, примитивы, разработка.
	
	Объект ВКР --- форматы хранения и обмена геометрическими и другими данными компьютерных чертежей.
	
	Цель работы --- разработать алгоритмы и программное обеспечение для чтения информации из файлов формата DXF, извлечения из них данных о геометрических примитивах и формирования файлов форматов TXT, SVG и JSON, содержащих всю необходимую информацию в удобном для последующей работы виде.
	
	Методы работы и исследования: теоретический анализ и последующий синтез информации, моделирование, разработка, тестирование.
	
	Результатом работы стали алгоритмы и конвертеры в виде программного обеспечения «primiview».
	
	Область применения разработанных алгоритмов и программного обеспечения: удобное использование файлов получаемых форматов данных при технологической разработке управляющих программ для производства изделий на станках с числовым программным управлением, а также при работе в CAD- и CAM-системах в целом.
	
	Практическая значимость работы заключается в использовании разработанных алгоритмов и программного обеспечения при разработке специализированных систем автоматизированного проектирования (САПР) и работе в них, а также, выполнении работ по хозяйственному договору \mbox{УрФУ} с ООО "Униматик".
	
	
	
\end{abstract}

%%% Local Variables: 
%%% mode: latex
%%% TeX-master: "rpz"
%%% End: 