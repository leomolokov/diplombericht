% Также можно использовать \Referat, как в оригинале
\begin{abstract}
	
Отчёт содержит \formbytotal{page}{страниц}{у}{ы}{}, \formbytotal{totalcount@figure}{рисун}{ок}{ка}{ков}, \formbytotal{totalcount@table}{таблиц}{у}{ы}{}, \ref{LastNumberedEquation} формул, \formbytotal{totalappendix}{приложен}{ия}{ий}{}.



%g2-105 package - why doesnt work!?!?!?
%\totalpages
%
%\totaltables
%\totalfigures
%\totalbibs  
%\pageref{reftotalpage}

Ключевые слова: коневертер, алгоритмы, программное обеспечение, примитивы, разработка.

Объект ВКР --- форматы хранения данных и представления графической информации.

Цель работы --- разработать программное обеспечение для дешифрования информации из файлов типа DXF, извлечении из них информации о геометрических примитивах и формирования файлов других форматов, содержащих всю необходимую информацию в удобном для последующей работы виде.

Методы работы и исследования: теоретический анализ и последующий синтез информации, моделирование, разработка, тестирование.

Результатом работы стали алгоритмы и программное обеспечение --- конвертер «DXF Primiview».

Область применения разработанных алгоритмов и программного обеспечения --- удобное использование получаемых форматов данных при технологической разработке управляющих программ для производства изделий на станках с числовым программным управлением.

Значимость работы заключается в использовании разработанных алгоритмов и программного обеспечения при разработке специальных САПР и работе в них, а также, выполнение работ по хозяйственному договору УрФУ с ООО "Униматик".



\end{abstract}

%%% Local Variables: 
%%% mode: latex
%%% TeX-master: "rpz"
%%% End: 
